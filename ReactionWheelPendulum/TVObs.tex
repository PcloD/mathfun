We consider the system 
\[
	\dot{x}_p = A_p(q)x_p + \beta_p(t), \ y=C_px_p,
\]
where
\[
	\begin{aligned}
		A_p(q) := \begin{bmatrix} 0 & 1 &  0\\ 0 & 0 & -a_1\cos(q)\\ 0 & 0 & 0 \end{bmatrix}, \ C_p := \begin{bmatrix} 1 & 0 & 1 \end{bmatrix}
	\end{aligned}
\]
with the known constant parameter $a_1>0$, measured external signal $q(t)$, and known reference input $\beta_p(t)$. The system is observable for $\cos(q)\ne 0$ and is not observable for $\cos(q)=0$. The goal is to design an observer for $x_p$.

First we define the unitary matrix
\[
	T_p :=  \begin{bmatrix} \frac{\sqrt{2}}{2} & 0 & -\frac{\sqrt{2}}{2}\\ 0 & 1 & 0\\ \frac{\sqrt{2}}{2} & 0 & \frac{\sqrt{2}}{2} 	 \end{bmatrix}
\]
and the change of coordinates $x := T_p x_p$. Hence
\[
	\dot{x} = A(q)x + \beta, \ y=Cx,
\]
where
\[
	\begin{aligned}
		A(q) &:= T_p A_p(q) T_p^\top = 
		\begin{bmatrix} 0 & \frac{\sqrt{2}}{2} & 0\\ \frac{\sqrt{2}}{2}a_{1}\cos(q) & 0 & -\frac{\sqrt{2}}{2}a_{1}\cos(q)\\ 0 & \frac{\sqrt{2}}{2} & 0 \end{bmatrix}, \\
		C &:= C_p T_p^\top = \begin{bmatrix} 0 & 0 & \sqrt{2} \end{bmatrix}, \ \beta:=T_p\beta_p.
	\end{aligned}
\]
When $\cos(q)\ne 0$ the pair $\left(A(q),C\right)$ is obviously observable. However, for $\cos(q)$ the matrix $A$ is in the stair-case observable form, \emph{i.e.} the states $x_{2}$ and $x_{3}$ remain observable while the state $x_{1}$ is not. Note that since the matrix $T_p$ is constant and invertible, the problem of estimation of $x_p$ is equivalent to the problem of estimation of $x$.


The characteristic polynomial of the matrix $A(q)$ is 
\[
	\det\left(sI-A(q)\right) \equiv s^3,
\]
and the canonical observable form of the system is given by
\[
	A_o := \begin{bmatrix} 0 & 0 & 0\\ 1 & 0 & 0\\ 0 & 1 & 0  \end{bmatrix}, \ C_o := \begin{bmatrix} 0 & 0 & 1 \end{bmatrix}.
\]
Define the change of coordinates $z:=T(q)x$, where
\[
	T(q) := \begin{bmatrix} C_o \\ C_oA_o \\ C_oA_o^2 \end{bmatrix}^{-1}
			  \begin{bmatrix} C \\ CA \\ CA^2 \end{bmatrix}  =
			  \begin{bmatrix} \frac{\sqrt{2}}{2}a_{1}\cos(q) & 0 & -\frac{\sqrt{2}}{2}a_{1}\cos(q)\\ 0 & 1 & 0\\ 0 & 0 & \sqrt{2} \end{bmatrix}.
\]
Then 
\[
	\dot{z} = A_o z + T(q)\beta, \ y= C_oz,
\]
and 
\[
	A_oT(q) = T(q)A(q), \ C_oT(q) = C.
\]
Note that the matrix $T(q)$ is singular for $\cos(q)=0$, and for $\cos(q)\ne0$ its inversion is given by
\[
	T^{-1}(q) = \begin{bmatrix} \frac{\sqrt{2}}{a_{1}\cos(q)} & 0 & \frac{\sqrt{2}}{2}\\ 0 & 1 & 0\\ 0 & 0 & \frac{\sqrt{2}}{2} \end{bmatrix}.
\] 

\noindent\emph{The first attempt}

For the state $z$ an observer can be designed as
\[
	\dot{\hat{z}} = A_o \hat{z} + T(q) \beta -L_o\tilde{y},
\]
where for all variables $\tilde{\left(\cdot\right)}:=\hat{\left(\cdot\right)}-\left(\cdot\right)$, and the matrix $L_o$ is such that for the matrix
\[
	F_o:=A_o -L_o C_o
\]
there exist matrices $P>0$ and $Q>0$ such that
\[
	F_o^\top P + PF_o^\top = -Q.
\]
The observation error dynamics is
\[
	\dot{\tilde{z}} = F_o \tilde{z}
\]
and for the Lyapunov function $V_o(\tilde{z}):=\tilde{z}^\top P \tilde{z}$ we have
\[
	\dot{V}_o = -\tilde{z}^\top Q\tilde{z}.
\]

Applying the inverse change of coordinates, $x=T^{-1}(q)z$, the corresponding observer for $x$ is given by
\begin{equation} \label{eq:ObsX}
	\dot{\hat{x}} = A(q)\hat{x} + \beta - L(q)\tilde{y},
\end{equation}
where $L(q):= T^{-1}(q)L_o$. Unfortunately, this inverse change of coordinates does not exist for $\cos(q)=0$, and, consequently, the vector of gains $L(q)$ is not defined for these values of $q$.

What happens with the Lyapunov function? For the same function $V_o$ we have
\[
	\begin{aligned}
		V_o &= \tilde{z}^\top P \tilde{z} = \tilde{x}^\top T^\top(q)PT(q) \tilde{x}, \\
		\dot{V}_o &= - \tilde{z}^\top Q \tilde{z} = -\tilde{x}^\top T^\top(q)QT(q) \tilde{x},
	\end{aligned}
\]
and we still have $\dot{V}_o<0$ for all $|\tilde{x}|\ne 0$ and $\cos{q}\ne 0$, but we cannot use this function to analyze $\tilde{x}$ since the matrix $T^\top(q)PT(q)$ is not positive definite for $\cos(q)=0$ and $V_o$ is not a proper Lyapunov function for $\tilde{x}$. Moreover, the observer \eqref{eq:ObsX} is still undefined for $\cos(q)=0$.

\bigskip

\noindent\emph{The second attempt}

Choose sufficiently small $0<\delta<1$ and define two sets:
\[
	\mathcal{Q}_+ := \left\{\,q \mid |\cos(q)|\ge \delta \,\right\}, \ \mathcal{Q}_0 := \left\{\,q \mid |\cos(q)|< \delta \,\right\}.
\]
Redefine the gains vector $L(q)$ in \eqref{eq:ObsX} as 
\begin{equation} \label{eq:newL}
	L(q) := \begin{bmatrix} \frac{\sqrt{2}}{2}l_{o3} \\ l_{o2} \\  \frac{\sqrt{2}}s{2}l_{o3}\end{bmatrix} + 
			\operatorname{ind}(q)\begin{bmatrix} \frac{\sqrt{2}}{a_1}l_{o1} \\0 \\0 \end{bmatrix} 
			= L_1 + \operatorname{ind}(q)L_2,
\end{equation}
where $l_{oi}$ is the $i$-th component of the vector $L_o$ and the indicator function $\operatorname{ind}(q)$ is defined as
\[
	\operatorname{ind}(q) := \begin{cases}
	\frac{1}{\cos(q)} \text{ for } q\in\mathcal{Q}_+, \\
	0 \text{ for } q\in\mathcal{Q}_0.
	\end{cases}
\]
For the coordinates $z$ we have
\[
	\dot{\tilde{z}} = A_o\tilde{z} -\begin{bmatrix} 0 \\ l_{o2} \\ l_{03} \end{bmatrix}C_o - \operatorname{ind}(q)\begin{bmatrix} l_{o1}\cos(q) \\0 \\ 0\end{bmatrix}C_o.
\]
Then 
\begin{itemize}
	\item for $q\in\mathcal{Q}_+$ it holds $L(q) = T^{-1}(q)L_0$ and 
	\[
		\begin{aligned}
			\dot{\tilde{x}} &= \left(A(q)-L(q)C\right) \tilde{x} = T^{-1}(q) F_o T(q)\tilde{x}, \\
			\dot{\tilde{z}} &= F_o \tilde{z}.
		\end{aligned}
	\]
	\item for $q\in\mathcal{Q}_0$ we have $L(q)=L_1$ and
	\[
		\begin{aligned}
			\dot{\tilde{x}} &= \left(A(q)-L_1C\right) \tilde{x}, \\
			\dot{\tilde{z}} &= F_1 \tilde{x},
		\end{aligned}
	\]
	where 
	\[
		F_1 := A_o - \begin{bmatrix} 0 \\ l_{o2} \\ l_{03} \end{bmatrix}C_o.
	\]
\end{itemize}



Define the Lyapunov function
\[
	V(\tilde{x}) = \tilde{x}^\top \left(T^\top(q)PT(q) + \gamma I\right)\tilde{x} = 
	\tilde{z}^\top P\tilde{z} + \gamma\tilde{x}^\top\tilde{x},
\]
where $\gamma>0$. There exist $\alpha_1>0$ and $\alpha_2>0$ such that
\[
	\alpha_1|\tilde{x}|^2 \le V \le \alpha_2|\tilde{x}|^2.
\]

Consider the case $q\in\mathcal{Q}_+$. Then the time derivative of $V$ is given by
\[
	\dot{V}_+ = - \tilde{x}^\top \left(T^\top(q) Q T(q) + \gamma R_+(q)\right) \tilde{x},
\]
where $R_+(q):=T^{-1}(q) F_o T(q) + T^\top(q) F_o^\top T^{-\top}(q)$. Let $\lambda_{min}(Q)$ be the smallest eigenvalue of $Q$ and $\lambda_{max}(R_+)$ be the largest eigenvalue of $R_+(q)$ over $\mathcal{Q}_+$. Choose 
\[
	\gamma = \frac{\lambda_{min}(Q)\sigma_{min}^2(T)}{2\lambda_{max}(R_+)},
\]
where $\sigma_{min_+}(T)$ is the smallest singular value of $T(q)$ over $\mathcal{Q}_+$ given by
\[
	\sigma_{min_+}^2 = \min_{q\in \mathcal{Q}_+}\lambda_{min}(T^\top(q)T(q)) = 1 + \frac{a_1^2\delta^2}{2} - \frac{\sqrt{a_1^4\delta^4+4}}{2} >0.
\]
Then we obtain
\[
	\dot{V}_+ \le - \frac{\lambda_{min}(Q)\sigma_{min}^2(T)}{2}|\tilde{x}|^2 \le -\frac{\lambda_{min}(Q)\sigma_{min}^2(T)}{2\alpha_2}V = -\kappa_1V.
\]

Next consider the case $q\in\mathcal{Q}_0$. The time derivative of $V$ is given by
\[
	\dot{V}_0 = \tilde{x}^\top\left( T^\top(q) \left(F_1^\top P + PF_1\right) T(q) + \gamma  R_0(q)\right)\tilde{x} =: \tilde{x}^\top \overline{R}_0(q) \tilde{x},
\]
where $R_0(q):=\left(A(q)-L_1C\right)^\top + \left(A(q)-L_1C\right)$. Let $\lambda_{max}(\overline{R}_0(q))$ be the largest eigenvalue of $\overline{R}_0(q)$ over $\mathcal{Q}_0$. Then
\[
	\dot{V}_0 \le \lambda_{max}(\overline{R}_0(q)) |\tilde{x}|^2 \le \frac{\lambda_{max}(\overline{R}_0(q))}{\alpha_1}V= \kappa_2V.
\]

To summarize, we have
\[
	\begin{cases}
		\dot{V} \le -\kappa_1V \text{ for } q\in \mathcal{Q}_+, \\
		\dot{V} \le \kappa_2V \text{ for } q\in \mathcal{Q}_0. \\
	\end{cases}
\]
Next we make the following assumption.

\noindent\emph{Assumption}. For the trajectory $q(t)$ there exists $T_q>0$ such that for all $t_0>0$
\begin{itemize}
	\item during the time interval $[t_0 \ t_0+T_q]$ the trajectory $q(t)$ travels between the sets $\mathcal{Q}_+$ and $\mathcal{Q}_0$ finite number of times;
	\item during the time interval $[t_0 \ t_0+T_q]$ the duration of time when $q\in\mathcal{Q}_+$ defined as $T_{q_+}$ and the duration of time when $q\in\mathcal{Q}_0$ defined as $T_{q_0}$ are such that
	\[
		\kappa_2T_{q_0} - \kappa_1T_{q_+} \le -\kappa_3T_q
	\]
	for some $\kappa_3>0$, where $T_{q_0} + T_{q_+} = T_q$.
\end{itemize}

With this assumption for any $t_0$ we have
\[
	V(t_0+T_q) \le \exp(-\kappa_1T_{q_+})\exp(\kappa_2T_{q_0})V(t_0) \le e^{-\kappa_3T_q}V(t_0)
\]
and 
\[
	V(t) \le e^{\kappa_2T_{q_0}}V(t_0) \ \forall t\ge t_0.
\]
Thus, $V(t) \to 0$.

\bigskip

\noindent\emph{The third attempt}

We can also consider a finite-time solution valid for $q\in\mathcal{Q}_+$ add assume that the trajectory spends enough time outside of $\mathcal{Q}_0$.



