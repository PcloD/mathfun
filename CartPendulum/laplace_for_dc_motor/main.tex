\documentclass{article}
\usepackage[utf8]{inputenc}
\usepackage[T2A]{fontenc}
\usepackage[russian]{babel}
\usepackage{graphicx}
\usepackage{amsmath,amssymb}


\title{Laplace for dc motor}
\author{ha}
\date{October 2016}

\newcommand{\Fourier}[1]{\mathcal{F}\left\{#1\right\}(s)}
\newcommand{\Laplace}[1]{\mathcal{L}\left\{#1\right\}(s)}
\newcommand{\ILaplace}[1]{\mathcal{L}^{-1}\left\{#1\right\}(t)}


\begin{document}

\maketitle

\section{Fourier transform}
$$
f(t) = 7\sin(11 t) + 19 \cos(41 t)
$$

$$
\Fourier{f(t)} : \mathbb{R} \rightarrow \mathbb{C}
$$

$$
\mathcal{F}\left\{f(t)\right\}(11) = 7i, ~~~ \mathcal{F}\left\{f(t)\right\}(19) = 41
$$

$$
f'(t) = 7 \cdot 11 \sin(11t+90^\circ) +  19\cdot 41\cos(41t+90^\circ) 
$$

$$
\Fourier{f'} = i s \Fourier{f}
$$

\section{Laplace transform}

$$
    \Laplace{af(t) + bg(t)} = a\Laplace{f(t)} + b\Laplace{g(t)}
$$

$$
    \Laplace{\frac{df}{dt}(t)} = s\Laplace{f(t)} - f(0)
$$

\section{Measure motor parameters}

Let us fix the motor shaft; then $\omega(t)\equiv 0$.
So if we provide $U(t)$ tension for our DC motor, then the current will follow next differential equation:

\begin{equation}
    L \frac{dI}{dt}(t) + RI(t) = U(t)
\label{eq:motor}
\end{equation}

The idea is to feed known $U(t)$, measure $I(t)$ and then deduce $R$ and $L$ from measurements.

\subsection{Unit step tension}

Initial conditions: $I(0) = 0$. Input tension is constant: $U(t) = U_0$.
Let us apply Laplace transform to both parts of our differential equation~\eqref{eq:motor}. 

\begin{align*}
    \Laplace{L \frac{dI}{dt}(t) + RI(t)} & = \Laplace{U(t)} \\
    L\Laplace{\frac{dI}{dt}(t)} + R\Laplace{I(t)} & = \frac{U_0}{s} \\
    L(s\Laplace{I(t)} - I(0)) + R\Laplace{I(t)} & = \frac{U_0}{s} \\
    Ls\Laplace{I(t)} + R\Laplace{I(t)} & = \frac{U_0}{s} \\
    (Ls+R)\Laplace{I(t)}  & = \frac{U_0}{s} \\
    \Laplace{I(t)}  & = \frac{U_0}{s(Ls+R)} \\
    \Laplace{I(t)}  & = \frac{U_0/R}{s} + \frac{-U_0/R}{s+R/L}
\end{align*}

The last equation is obtained with the method of partial fractions.
Now let us apply the inverse Laplace transform:
\begin{align*}
\ILaplace{\Laplace{I(t)}}  & = \ILaplace{\frac{U_0/R}{s} + \frac{-U_0/R}{s+R/L}} \\
I(t)  & = \frac{U_0}{R}\ILaplace{\frac{1}{s}} - \frac{U_0}{R}\ILaplace{\frac{1}{s+R/L}}\\
I(t) & = \frac{U_0}{R} - \frac{U_0}{R}e^{-t R/L}
\end{align*}

It makes perfect sense: in several milliseconds from beginning the current will be equal to $U_0/R$ (Ohm's law); and the time necessary to reach this current depends directly on the inductance $L$.

\subsection{Sinusoidal tension}
Initial conditions: $I(0) = 0$. Input tension is sinusoidal: $U(t) = U_0\sin(F_0 t)$.
As in the previous section, let us apply Laplace transform to both parts of our differential equation~\eqref{eq:motor}. 
Our input tension is sinusoidal, so:
$$
\Laplace{U_0\sin(F_0t)} = \frac{U_0 F_0}{s^2 + F_0^2}
$$

then it is easy to see that
\begin{align*}
    \Laplace{I(t)}  & = \frac{F_0}{(s^2 + F_0^2)(Ls+R)} \\
    \Laplace{I(t)}  & = \frac{-L U_0 F_0}{L^2 F_0^2 + R^2}\frac{s}{s^2 + F_0^2} +  \frac{R U_0 F_0}{L^2 F_0^2 + R^2}\frac{1}{s^2 + F_0^2} + \frac{L U_0 F_0}{L^2 F_0^2 + R^2}\frac{1}{s + R/L}  \\
\end{align*}
As before, apply the inverse Laplace transform and get the output signal:
$$
I(t) = \frac{-L U_0 F_0}{L^2 F_0^2 + R^2}\cos(F_0t) +  \frac{R U_0}{L^2 F_0 + R^2}\sin(F_0 t) + \frac{L U_0 F_0}{L^2 F_0^2 + R^2}e^{-tR/L} 
$$
And again it makes sense: after several milliseconds current will be a sinusoidal signal of the same frequency as the input tension, however it will have a lag (sum of a sine and a cosine is a sine with a lag). And at the very beginning the current is zero. Sanity check passed.

% http://stackoverflow.com/questions/12233702/fitting-transfer-function-models-in-scipy-signal
% https://courses.engr.illinois.edu/ece486/documents/set5.pdf

\newpage

$$
\tau \frac{dI(t)}{dt} + I(t) = J(t)
$$

$$
(\tau s + 1)\Laplace{I(t)}  = \Laplace{J(t)}
$$

\begin{equation}
\frac{\Laplace{I(t)}}{\Laplace{J(t)}} = \frac{1}{\tau s + 1}
\label{eq:currentloop}
\end{equation}

$$
E(t) = J(t) - I(T)
$$

$$
\omega(t) \equiv 0
$$

$$
\frac{\Laplace{I(t)}}{\Laplace{U(t)}} = \frac{1}{Ls+R}
$$

\begin{align*}
    \frac{\Laplace{I(t)}}{\Laplace{U(t)}} \cdot \frac{\Laplace{J(t)}}{\Laplace{I(t)}} & = \frac{1}{Ls+R}  \cdot \frac{\tau s + 1}{1}\\
    \frac{\Laplace{E(t) + I(t)}}{\Laplace{U(t)}}  & = \frac{\tau s+1}{Ls+R}\\
    \frac{\Laplace{E(t)}}{\Laplace{U(t)}} + \frac{\Laplace{I(t)}}{\Laplace{U(t)}}  & = \frac{\tau s+1}{Ls+R}\\
    \frac{\Laplace{E(t)}}{\Laplace{U(t)}}   & = \frac{\tau s+1}{Ls+R} - \frac{1}{Ls+R}\\
    \frac{\Laplace{E(t)}}{\Laplace{U(t)}}   & = \frac{\tau s}{Ls+R}\\
\end{align*}

$$
\Laplace{U(t)} = \left(\frac{L}{\tau} + \frac{R}{\tau s}\right)\Laplace{E(t)}
$$

$$
\Laplace{\int_0^tf(v)dv} = \frac{1}{s}\Laplace{f(t)}
$$

$$
U(t) = \frac{L}{\tau}E(t) + \frac{R}{\tau}\int_0^t E(v) dv
$$

\section{Velocity control}

Having the current control law implemented, the current loop is given by
\[
    \tau \frac{d}{dt}I(t) + I(t) = I^\star(t),
\]
where $I^\star$ is the current reference. The acceleration is proportional to the torque:
\[
    \frac{d}{dt}v(t) = k_0(I(t) - I_{e}(t)),
\]
where $v$ is the cart velocity, and $I_{e}$ is the current equivalent of all the external forces. Therefore, the transfer functions from $I^\star$ and $I_e$ to $v$ are
%\[
%    W_{I^\star \to v}(s) = \frac{k_0}{(\tau s +1 )s}.
%\]
\[
    v(s) = \frac{k_0}{(\tau s +1 )s}I^\star(s) + \frac{k_0}{s}I_e(s).
\]
Define the velocity tracking error as $e_v(t):=v^\star(t)-v(t)$, where $v^\star$ is the velocity reference, and apply simple PI controller
\[
    I^\star(t) = k_p e_v(t) + k_i \int_0^t{e_v(\sigma)d\sigma}
\]
with the transfer function
\[
    I^\star(s) = \frac{k_ps + k_i}{s}e_v(s).
\]
The closed-loop transfer functions from $v^\star$ and $I_e$ to $v$ are thus
%\[
%    W_{v^\star \to v}(s) = \frac{k_0(k_ps + k_i)}{\tau s^3 + s^2 + k_0k_ps + k_0k_i}.
%\]
\[
    v(s) = \frac{k_0(k_ps + k_i)}{\tau s^3 + s^2 + k_0k_ps + k_0k_i} v^\star(s) + \frac{k_0s(\tau s + 1)}{\tau s^3 + s^2 + k_0k_ps + k_0k_i}I_e(s).
\]
Stability and performance of the system are defined by the denominator roots, so we wish to make them "good". We have only 2 design parameters, and we have 3 roots, $r_1$, $r_2$ and $r_3$, with the constrain 
\[
    r_1+r_2+r_3 = -\frac{1}{\tau}.
\]
So we choose $r_1$ and $r_2$, compute $r_3 = -\frac{1}{\tau}-r_1-r_2$ and construct the desired polynomial
\[
    d(s) = \tau(s-r_1)(s-r_2)(s-r_3)=\tau s^3+s^2+d_1s+d_0.
\]
Finally,
\[
    k_p = \frac{d_1}{k_0},\quad k_i=\frac{d_0}{k_0}.
\]

We choose
\[
    r_{1,2}=-20(\cos(\frac{\pi}{4}) \pm \sqrt{-1}\sin(\frac{\pi}{4}))
\]
and expect transient time around 0.2 second.

% tau:.002;
% k0:4;
% r1:-20*(cos(%pi/4)+%i*sin(%pi/4));
% r2:-20*(cos(%pi/4)-%i*sin(%pi/4));
% r3:-1/tau-r1-r2;
% float(radcan(tau*(s-r1)*(s-r2)*(s-r3),s));
% ki = 188686*.002/4 = 94.343
% kp = 13742*.002/4 = 6.871


% r1:-40*(cos(%pi/4)+%i*sin(%pi/4));
% r2:-40*(cos(%pi/4)-%i*sin(%pi/4));
% ki = 709490*.002/4 = 354.745
% kp = 26684*.002/4 = 13.342


If we take $\tau=.002$ and $k_0=4$, then 



\section{Friction}
Since we do not measure any force directly, and since the force we produce is (in our model) proportional to the current we measure, below we consider the friction as an equivalent current.

We use the simplified Columbus plus viscoгs friction model given by
\[
    I_{fr}(v) = F_c\mathrm{sign}(v) + F_v v,
\]
where $I_{fr}$ is the current equivalent to the friction force, $v$ is the velocity of the cart and $F_c>0$, $F_v>0$ are the constant parameters of the friction model.

Consider an experiment where the cart is moving with a constant velocity. Since acceleration is (almost) zero, we assume that in such an experiment the current we produce is equal to $I_{fr}$. Repeating such an experiment with different velocities we obtain a set of points $(v_i,I_{fr}(v_i))$. Given this set we can 1) check, if these points satisfy the simplified friction model; 2) estimate $F_c$ and $F_v$.

As soon as $F_c$ is estimated, we can add the friction compensation term to our control. Suppose that the desired current computed by a top-level pendulum controller is $I_{d}$. Then the current loop reference $I^\star$ is
\[
    I^\star = I_d + F_c\mathrm{sign}(v).
\]

\end{document}
